\chapter*{CHƯƠNG 3: XÂY DỰNG ỨNG DỤNG }
\addcontentsline{toc}{chapter}{CHƯƠNG 3: XÂY DỰNG ỨNG DỤNG }

\subsection*{\textbf{3.1 Công nghệ sử dụng}}
\addcontentsline{toc}{section}{3.1 Công nghệ sử dụng}

\subsection*{\textbf{3.1.1 Ngôn ngữ lập trình}}
\begin{itemize}
    \item Thiết kế giao diện với HTML, CSS, JavaScript, Bootstrap
    \item Xử lý logic bằng ngôn ngữ lập trình Node.js
    \item Quản trị cơ sở dữ liệu: MySQL
\end{itemize}

\subsection*{\textbf{3.1.2 Môi trường cài đặt}}    
\begin{itemize}
    \item Sử dụng phần mềm Visual Studio Code
    \item Server Node.js với Express.js
    \item Đẩy hệ thống lên ngrok
\end{itemize}

\subsection*{\textbf{3.2 Giao diện phần mềm}}
\addcontentsline{toc}{section}{3.2 Giao diện phần mềm}

\begin{figure}[H]
  \centering
  \includegraphics[width=0.8\textwidth]{gd1.png}
  \caption{Trang đăng nhập TimeLuxe}
  \label{fig:login}
\end{figure}

\begin{figure}[H]
  \centering
  \includegraphics[width=0.8\textwidth]{gd2.png}
  \caption{Trang đăng ký TimeLuxe}
  \label{fig:register}
\end{figure}

\begin{itemize}
    \item Giao diện phía quản trị viên
\end{itemize}

\begin{figure}[H]
  \centering
  \includegraphics[width=0.8\textwidth]{gd3.jpeg}
  \caption{Trang chủ quản trị viên TimeLuxe}
  \label{fig:admin-dashboard}
\end{figure}

\begin{figure}[H]
  \centering
  \includegraphics[width=0.8\textwidth]{gd4.jpeg}
  \caption{Trang báo cáo của quản trị viên TimeLuxe}
  \label{fig:admin-report}
\end{figure}

Hình 30 minh họa giao diện trang báo cáo và bảng điều khiển dành cho quản trị viên trong hệ thống TimeLuxe. Giao diện này cung cấp cái nhìn tổng thể về hoạt động kinh doanh đồng hồ thông qua các số liệu trực quan và biểu đồ thống kê. Tại đây, quản trị viên có thể theo dõi nhanh chóng các thông tin quan trọng như doanh thu bán đồng hồ, số lượng đơn hàng, tổng lợi nhuận, sản phẩm đồng hồ bán chạy, trạng thái đơn hàng, tỷ lệ khách hàng mới, cùng với các biểu đồ cột, biểu đồ tròn giúp phân tích dữ liệu hiệu quả hơn.

\begin{figure}[H]
  \centering
  \includegraphics[width=0.8\textwidth]{gd5.png}
  \caption{Trang quản lý tài khoản TimeLuxe}
  \label{fig:admin-accounts}
\end{figure}

Hình 31 thể hiện giao diện quản lý tài khoản trong hệ thống quản trị TimeLuxe dành cho quản trị viên (admin). Trang này cho phép hiển thị danh sách các tài khoản người dùng, bao gồm các thông tin cơ bản như: ID, tên người dùng, số điện thoại, email, vai trò, và các tùy chọn thao tác. Các vai trò được phân loại rõ ràng như: admin, employee, customer, giúp quản lý hệ thống dễ dàng phân quyền và kiểm soát truy cập.

Giao diện được thiết kế trực quan với bảng dữ liệu gọn gàng và dễ theo dõi. Ở phía bên phải của mỗi dòng tài khoản là các nút thao tác như chỉnh sửa (biểu tượng bút) và xóa (biểu tượng thùng rác), cho phép quản trị viên nhanh chóng cập nhật hoặc loại bỏ tài khoản khi cần thiết. Ngoài ra, góc trên bên phải còn có nút "Thêm Tài Khoản", hỗ trợ thêm mới người dùng vào hệ thống một cách linh hoạt.

\begin{figure}[H]
  \centering
  \includegraphics[width=0.8\textwidth]{gd6.png}
  \caption{Trang quản lý sản phẩm đồng hồ}
  \label{fig:admin-products}
\end{figure}

Hình 32 minh họa giao diện trang quản lý sản phẩm đồng hồ của quản trị viên trong hệ thống TimeLuxe. Trang này cho phép hiển thị danh sách các sản phẩm đồng hồ đang có trong hệ thống cùng với các thông tin chi tiết như: tên sản phẩm, giá bán, tình trạng hàng tồn kho, trạng thái hiển thị, ngày đăng, và các chức năng như chỉnh sửa hoặc xóa sản phẩm.

Giao diện được chia thành các dòng tương ứng với từng sản phẩm đồng hồ, kèm theo các biểu tượng màu sắc giúp quản trị viên dễ dàng nhận biết trạng thái như "Hiển thị", "Đang ẩn", "Hết hàng" hoặc "Sắp hết". Ngoài ra, hệ thống cũng hỗ trợ các thao tác nhanh như xem chi tiết, sửa đổi thông tin hoặc xóa sản phẩm khỏi danh sách thông qua các nút chức năng ở cuối mỗi dòng.

Thanh điều hướng bên trái cung cấp các mục liên quan như: Trang chính, Báo cáo, Tài khoản, Sản phẩm, Danh mục và Hàng tồn kho, giúp việc chuyển đổi giữa các phần quản lý trở nên linh hoạt và thuận tiện. Giao diện này hỗ trợ quản trị viên kiểm soát tốt danh mục sản phẩm đồng hồ, đảm bảo cập nhật thông tin kịp thời và chính xác phục vụ cho quá trình bán hàng.

\begin{figure}[H]
  \centering
  \includegraphics[width=0.8\textwidth]{gd7.png}
  \caption{Trang danh mục sản phẩm đồng hồ}
  \label{fig:admin-categories}
\end{figure}

\begin{figure}[H]
  \centering
  \includegraphics[width=0.8\textwidth]{gd8.png}
  \caption{Trang quản lý hãng đồng hồ}
  \label{fig:admin-brands}
\end{figure}

\begin{itemize}
    \item Giao diện phía nhân viên
\end{itemize}

\begin{figure}[H]
  \centering
  \includegraphics[width=0.8\textwidth]{gd9.png}
  \caption{Trang quản lý đơn hàng đồng hồ}
  \label{fig:employee-orders}
\end{figure}

Hình 35 thể hiện giao diện trang quản lý đơn hàng của nhân viên trong hệ thống TimeLuxe. Giao diện này cho phép người dùng (nhân viên) theo dõi và xử lý toàn bộ thông tin liên quan đến các đơn hàng đồng hồ đã phát sinh trong hệ thống. Mỗi dòng đơn hàng hiển thị rõ ràng mã đơn (#), số lượng sản phẩm đồng hồ, thông tin sản phẩm, giá trị đơn hàng, thời gian đặt hàng, cùng với trạng thái xử lý, trạng thái thanh toán và giao hàng.

Các nhãn màu như "Đang xử lý", "Đã hoàn thành", "Chờ giao hàng", "Chờ thanh toán" giúp phân loại trực quan tình trạng hiện tại của mỗi đơn, hỗ trợ nhân viên nhanh chóng đưa ra hành động phù hợp. Ngoài ra, mỗi đơn hàng còn có nút "Sửa" và "In" ở cột thao tác bên phải, cho phép chỉnh sửa thông tin đơn hàng hoặc in hóa đơn/biên nhận một cách nhanh chóng.

\begin{figure}[H]
  \centering
  \includegraphics[width=0.8\textwidth]{gd10.png}
  \caption{Trang quản lý tin tức}
  \label{fig:employee-news}
\end{figure}

\begin{figure}[H]
  \centering
  \includegraphics[width=0.8\textwidth]{gd11.png}
  \caption{Trang quản lý chat}
  \label{fig:employee-chat}
\end{figure}

Hình ảnh trên thể hiện giao diện Quản lý Chat hỗ trợ trong hệ thống TimeLuxe, dành cho nhân viên hỗ trợ khách hàng. Giao diện được chia thành hai phần chính: danh sách các phòng chat bên trái và nội dung trò chuyện chi tiết bên phải.

Phần danh sách phòng chat hiển thị các đoạn hội thoại đang hoạt động hoặc đã kết thúc, bao gồm tên khách hàng và thời gian bắt đầu cuộc trò chuyện. Mỗi phiên chat đều có trạng thái active (đang hoạt động) giúp nhân viên dễ dàng nhận biết và theo dõi tình trạng hỗ trợ.

Phần nội dung trò chuyện ở bên phải hiển thị chi tiết các tin nhắn trao đổi giữa nhân viên (ví dụ: employee1) và khách hàng (ví dụ: customer1). Nhân viên có thể chào hỏi, tiếp nhận và xử lý yêu cầu, trong khi khách hàng cũng có thể phản hồi trực tiếp tại ô nhập tin nhắn phía dưới. Ngoài ra, hệ thống còn cho phép kết thúc cuộc trò chuyện (nút "Đóng chat") khi phiên hỗ trợ hoàn tất.

\begin{figure}[H]
  \centering
  \includegraphics[width=0.8\textwidth]{gd12.png}
  \caption{Trang quản lý voucher}
  \label{fig:employee-voucher}
\end{figure}

Hình 38 minh họa giao diện quản lý voucher trong hệ thống TimeLuxe, dành cho nhân viên. Giao diện này bao gồm hai khu vực chính: thêm voucher mới và danh sách voucher hiện có.

Phần trên là form tạo mới voucher, nơi nhân viên có thể nhập đầy đủ thông tin như: mã voucher, mô tả, loại giảm giá như phần trăm hoặc số tiền, giá trị giảm, số lượng, điều kiện đơn tối thiểu, giới hạn giảm tối đa, cùng với ngày bắt đầu, ngày kết thúc và trạng thái hoạt động (kích hoạt hoặc chưa). Sau khi hoàn tất, người dùng có thể nhấn nút "Thêm mới" để lưu thông tin.

Bên dưới là bảng danh sách các voucher đã tạo, hiển thị chi tiết các thuộc tính: mã, mô tả, loại giảm giá, giá trị, điều kiện sử dụng, số lượng còn lại, thời gian áp dụng, trạng thái và các nút hành động (chỉnh sửa hoặc xóa). Trạng thái voucher được hiển thị bằng nhãn màu active, giúp người dùng dễ dàng nhận diện voucher nào đang được áp dụng.

\begin{itemize}
    \item Giao diện phía khách hàng
\end{itemize}

\begin{figure}[H]
  \centering
  \includegraphics[width=0.8\textwidth]{gd13.png}
  \caption{Trang chủ khách hàng TimeLuxe}
  \label{fig:customer-home}
\end{figure}

Hình 39 minh họa trang chủ của khách hàng trong hệ thống bán đồng hồ trực tuyến TimeLuxe. Giao diện được thiết kế hiện đại, thân thiện và dễ sử dụng, với bố cục chia thành nhiều khu vực rõ ràng, hỗ trợ khách hàng tìm kiếm sản phẩm đồng hồ và thông tin nhanh chóng.

Phần đầu trang là banner chính, nổi bật với thông điệp thương hiệu "TimeLuxe - Đồng Hồ Cao Cấp", kết hợp hình ảnh minh họa các sản phẩm đồng hồ và nút "Khám Phá Ngay" giúp điều hướng người dùng đến các danh mục sản phẩm nổi bật.

Ngay bên dưới là khu vực "Hàng mới về", hiển thị danh sách các sản phẩm đồng hồ mới cập nhật kèm theo hình ảnh, tên sản phẩm, giá bán và nhãn dán nổi bật như "Mới", "Sale". Điều này giúp thu hút sự chú ý của người dùng và khuyến khích hành vi mua sắm.

Phía dưới tiếp theo là mục "Tin tức & Xu hướng", nơi khách hàng có thể đọc thêm các bài viết hữu ích, cập nhật xu hướng đồng hồ hoặc thông báo khuyến mãi từ hệ thống.

Cuối trang là footer được trình bày đầy đủ thông tin: liên hệ, chính sách bảo mật, điều khoản sử dụng, mạng xã hội, cũng như hướng dẫn mua hàng và thông tin công ty. Tổng thể giao diện không chỉ giúp khách hàng dễ dàng trải nghiệm mua sắm mà còn thể hiện sự chuyên nghiệp, gần gũi và mang đậm bản sắc thương hiệu đồng hồ cao cấp.

\begin{figure}[H]
  \centering
  \includegraphics[width=0.8\textwidth]{gd14.png}
  \caption{Trang chi tiết sản phẩm đồng hồ}
  \label{fig:customer-product-detail}
\end{figure}

Hình 40 thể hiện giao diện trang chi tiết sản phẩm đồng hồ trong hệ thống bán hàng trực tuyến TimeLuxe, nơi người dùng có thể xem thông tin cụ thể của một sản phẩm đồng hồ trước khi quyết định mua. Trong ví dụ này là sản phẩm đồng hồ, được trình bày với hình ảnh lớn cùng các hình ảnh phụ mô tả chi tiết từng góc cạnh và tính năng.

Phía bên phải là phần mô tả ngắn gọn bao gồm: tên sản phẩm đồng hồ, giá bán, trạng thái còn hàng, số lượng có thể chọn, nút "Thêm vào giỏ hàng", và các thông tin như thương hiệu, đánh giá (số sao), số lượng bán ra. Dưới đó là các tab chuyển đổi như Mô tả sản phẩm, Thông tin bổ sung, Đánh giá, cho phép người dùng tìm hiểu kỹ hơn về sản phẩm.

Bên dưới là phần "Sản phẩm liên quan", đề xuất các mặt hàng đồng hồ tương tự hoặc cùng danh mục, giúp người dùng có thêm lựa chọn thay thế hoặc bổ sung khi mua sắm.

\begin{figure}[H]
  \centering
  \includegraphics[width=0.8\textwidth]{gd15.png}
  \caption{Giới thiệu cửa hàng TimeLuxe}
  \label{fig:customer-about}
\end{figure}

\begin{figure}[H]
  \centering
  \includegraphics[width=0.8\textwidth]{gd16.png}
  \caption{Trang tin tức}
  \label{fig:customer-news}
\end{figure}

\begin{figure}[H]
  \centering
  \includegraphics[width=0.8\textwidth]{gd17.png}
  \caption{Trang giỏ hàng}
  \label{fig:customer-cart}
\end{figure}

Hình 43 mô tả giao diện trang giỏ hàng của khách hàng trong hệ thống bán đồng hồ trực tuyến TimeLuxe. Giao diện này cho phép người dùng quản lý các sản phẩm đồng hồ đã thêm vào giỏ hàng trước khi tiến hành thanh toán.

Phía bên trái là danh sách sản phẩm đồng hồ trong giỏ, hiển thị rõ tên sản phẩm, hình ảnh, đơn giá, số lượng, và thành tiền tương ứng. Người dùng có thể điều chỉnh số lượng sản phẩm bằng các nút tăng/giảm hoặc xóa sản phẩm đã chọn. Bên trên có tùy chọn chọn tất cả và xóa sản phẩm đã chọn, giúp thao tác nhanh chóng với nhiều mặt hàng.

Phía bên phải là bảng tổng hợp đơn hàng, hiển thị tạm tính, số lượng sản phẩm đã chọn, phí vận chuyển, thuế VAT, lựa chọn và áp dụng mã giảm giá (voucher). Tổng chi phí được tính toán lại sau khi áp dụng khuyến mãi. Nút "Tiến hành thanh toán" cho phép khách hàng chuyển sang bước tiếp theo trong quá trình đặt hàng, trong khi nút "Tiếp tục mua sắm" hỗ trợ quay lại trang sản phẩm.

\begin{figure}[H]
  \centering
  \includegraphics[width=0.8\textwidth]{gd18.png}
  \caption{Trang đơn hàng}
  \label{fig:customer-orders}
\end{figure}

\subsection*{\textbf{3.3 Kịch bản kiểm thử hệ thống}}
\addcontentsline{toc}{section}{3.3 Kịch bản kiểm thử hệ thống}

\begin{longtable}{|p{0.8cm}|p{5cm}|p{8cm}|}
\hline
\textbf{STT} & \textbf{Chức năng} & \textbf{Kịch bản kiểm thử} \\
\hline
1 & Đăng nhập & Kiểm tra chức năng đăng nhập với tài khoản hợp lệ và không hợp lệ, kiểm tra phân quyền sau khi đăng nhập. \\
\hline
2 & Trang dashboard (admin) & Kiểm tra xem các thông tin báo cáo (doanh thu, sản phẩm đồng hồ, người dùng, biểu đồ) có hiển thị đúng sau khi đăng nhập không. \\
\hline
3 & Quản lý sản phẩm đồng hồ & Kiểm tra khả năng thêm mới, sửa thông tin, xóa và xem danh sách sản phẩm đồng hồ. \\
\hline
4 & Quản lý tài khoản & Kiểm tra thêm, sửa, xóa tài khoản người dùng; phân quyền và hiển thị đúng vai trò. \\
\hline
5 & Quản lý đơn hàng (nhân viên) & Kiểm tra hiển thị danh sách đơn hàng, xem trạng thái, in đơn hàng, và thay đổi trạng thái đơn. \\
\hline
6 & Chat hỗ trợ & Kiểm tra việc gửi, nhận tin nhắn giữa nhân viên và khách hàng, đóng phiên chat. \\
\hline
7 & Quản lý voucher & Kiểm tra thêm mới, sửa, xóa voucher; áp dụng voucher trong đơn hàng. \\
\hline
8 & Trang chủ (khách hàng) & Kiểm tra hiển thị banner, danh sách sản phẩm đồng hồ mới về, tin tức và xu hướng. \\
\hline
9 & Chi tiết sản phẩm đồng hồ & Kiểm tra thông tin sản phẩm, hình ảnh, đánh giá, số lượng tồn kho và khả năng thêm vào giỏ hàng. \\
\hline
10 & Giỏ hàng & Kiểm tra cập nhật số lượng sản phẩm, xóa sản phẩm, áp dụng mã giảm giá và tính tổng đơn hàng. \\
\hline
11 & Thanh toán & Kiểm tra tiến trình đặt hàng, xác nhận đơn hàng, xử lý thanh toán và trạng thái đơn sau khi đặt. \\
\hline
\end{longtable}

\subsection*{\textbf{3.4 Kiểm thử hệ thống}}
\addcontentsline{toc}{section}{3.4 Kiểm thử hệ thống}

\section*{Test Case – Chức năng Đăng nhập}

\begin{longtable}{|p{1cm}|p{3cm}|p{4cm}|p{3cm}|p{2cm}|}
\hline
\textbf{TC} & \textbf{Tên kiểm thử} & \textbf{Mô tả kiểm thử} & \textbf{Kết quả mong đợi} & \textbf{Trạng thái} \\
\hline
TC01 & Đăng nhập hợp lệ (admin) & Nhập đúng tài khoản và mật khẩu của quản trị viên & Hệ thống chuyển đến giao diện dashboard admin & Pass \\
\hline
TC02 & Đăng nhập hợp lệ (employee) & Nhập đúng tài khoản nhân viên & Chuyển đến giao diện quản lý đơn hàng dành cho nhân viên & Pass \\
\hline
TC03 & Đăng nhập hợp lệ (customer) & Nhập đúng tài khoản khách hàng & Chuyển đến trang chủ dành cho khách hàng & Pass \\
\hline
TC04 & Sai mật khẩu & Nhập tài khoản đúng nhưng sai mật khẩu & Hiển thị thông báo lỗi "Sai tài khoản hoặc mật khẩu" & Pass \\
\hline
TC05 & Tài khoản không tồn tại & Nhập tên đăng nhập không tồn tại trong hệ thống & Hiển thị thông báo "Tài khoản không tồn tại" & Pass \\
\hline
TC06 & Tài khoản bị vô hiệu hóa & Nhập đúng tài khoản đã bị khóa & Hiển thị thông báo "Tài khoản của bạn đã bị khóa" & Pass \\
\hline
TC07 & Trống tài khoản/mật khẩu & Bỏ trống một trong hai trường thông tin & Hiển thị thông báo yêu cầu nhập đầy đủ thông tin & Pass \\
\hline
TC08 & Kiểm tra phân quyền sau đăng nhập & Đăng nhập bằng tài khoản customer, cố gắng truy cập dashboard admin qua URL & Hệ thống hiển thị lỗi hoặc chuyển hướng về trang chính & Pass \\
\hline
\end{longtable}

\section*{Test Case – Kiểm tra hiển thị thông tin báo cáo (Dashboard)}

\begin{longtable}{|p{1cm}|p{3cm}|p{4cm}|p{3cm}|p{2cm}|}
\hline
\textbf{TC} & \textbf{Tên kiểm thử} & \textbf{Mô tả kiểm thử} & \textbf{Kết quả mong đợi} & \textbf{Trạng thái} \\
\hline
TC09 & Hiển thị doanh thu & Truy cập trang dashboard sau đăng nhập và kiểm tra mục "Tổng doanh thu" & Tổng doanh thu được hiển thị đúng theo dữ liệu trong hệ thống & Pass \\
\hline
TC10 & Hiển thị tổng số sản phẩm đồng hồ & Kiểm tra ô thống kê số lượng sản phẩm đồng hồ đang có & Hiển thị đúng số sản phẩm đồng hồ hiện có trong kho & Pass \\
\hline
TC11 & Hiển thị tổng số người dùng & Kiểm tra mục thống kê tổng số người dùng đã đăng ký & Hiển thị chính xác số lượng tài khoản người dùng & Pass \\
\hline
TC12 & Biểu đồ doanh thu theo thời gian & Kiểm tra biểu đồ doanh thu trên dashboard theo từng tháng & Biểu đồ hiển thị chính xác số liệu và cột thời gian đúng định dạng & Pass \\
\hline
TC13 & Biểu đồ số lượng đơn hàng & Kiểm tra biểu đồ đơn hàng theo từng trạng thái & Các phần biểu đồ phản ánh đúng trạng thái: hoàn thành, chờ xử lý, đã hủy,... & Pass \\
\hline
TC14 & Kiểm tra khả năng làm mới dữ liệu & Làm mới (refresh) dashboard sau khi có dữ liệu mới phát sinh & Các thống kê và biểu đồ được cập nhật kịp thời & Pass \\
\hline
\end{longtable}

\section*{Test Case – Chức năng Quản lý sản phẩm đồng hồ}

\begin{longtable}{|p{1cm}|p{3cm}|p{4cm}|p{3cm}|p{2cm}|}
\hline
\textbf{TC} & \textbf{Tên kiểm thử} & \textbf{Mô tả kiểm thử} & \textbf{Kết quả mong đợi} & \textbf{Trạng thái} \\
\hline
TC15 & Thêm sản phẩm đồng hồ mới – hợp lệ & Nhập đầy đủ thông tin sản phẩm đồng hồ và nhấn "Lưu" & Sản phẩm đồng hồ mới xuất hiện trong danh sách sản phẩm & Pass \\
\hline
TC16 & Thêm sản phẩm – thiếu trường bắt buộc & Bỏ trống tên hoặc giá khi thêm sản phẩm đồng hồ & Hệ thống hiển thị thông báo lỗi và không cho lưu & Pass \\
\hline
TC17 & Sửa thông tin sản phẩm đồng hồ & Nhấn nút "Chỉnh sửa" và thay đổi giá sản phẩm đồng hồ & Thông tin mới được cập nhật đúng trong danh sách & Pass \\
\hline
TC18 & Xóa sản phẩm đồng hồ & Nhấn nút "Xóa" trên dòng sản phẩm đồng hồ cần xóa & Sản phẩm đồng hồ bị loại bỏ khỏi danh sách hiển thị & Pass \\
\hline
TC19 & Xem danh sách sản phẩm đồng hồ & Truy cập mục "Sản phẩm" từ thanh menu & Danh sách sản phẩm đồng hồ hiển thị đầy đủ và đúng dữ liệu & Pass \\
\hline
TC20 & Tìm kiếm sản phẩm đồng hồ theo tên & Nhập từ khóa tìm kiếm vào ô tìm kiếm & Chỉ hiển thị sản phẩm đồng hồ có tên khớp với từ khóa & Pass \\
\hline
TC21 & Hiển thị trạng thái sản phẩm đồng hồ & Kiểm tra nhãn trạng thái "Hiển thị", "Ẩn", "Hết hàng" trong danh sách & Mỗi sản phẩm đồng hồ có trạng thái rõ ràng và đúng logic tồn kho & Pass \\
\hline
\end{longtable}

\section*{Test Case – Chức năng Quản lý tài khoản người dùng}

\begin{longtable}{|p{1cm}|p{3cm}|p{4cm}|p{3cm}|p{2cm}|}
\hline
\textbf{TC} & \textbf{Tên kiểm thử} & \textbf{Mô tả kiểm thử} & \textbf{Kết quả mong đợi} & \textbf{Trạng thái} \\
\hline
TC22 & Thêm tài khoản hợp lệ & Nhập đầy đủ họ tên, email, mật khẩu, vai trò và nhấn "Lưu" & Tài khoản hiển thị trong danh sách với vai trò đúng & Pass \\
\hline
TC23 & Thêm tài khoản thiếu thông tin & Bỏ trống email hoặc mật khẩu khi thêm & Hệ thống hiển thị lỗi "Vui lòng nhập đầy đủ thông tin" & Pass \\
\hline
TC24 & Sửa thông tin tài khoản & Nhấn "Chỉnh sửa", thay đổi tên người dùng & Thông tin được cập nhật đúng trong danh sách & Pass \\
\hline
TC25 & Đổi vai trò tài khoản & Thay đổi vai trò từ "employee" sang "admin" & Vai trò được cập nhật đúng và hiển thị chính xác & Pass \\
\hline
TC26 & Xóa tài khoản & Nhấn "Xóa" tài khoản bất kỳ trong danh sách & Tài khoản biến mất khỏi danh sách người dùng & Pass \\
\hline
TC27 & Hiển thị đúng vai trò người dùng & Mỗi tài khoản trong bảng hiển thị đúng vai trò: admin, employee, customer & Vai trò hiển thị rõ ràng, chính xác & Pass \\
\hline
TC28 & Ngăn xóa tài khoản đang hoạt động & Cố gắng xóa tài khoản admin đang đăng nhập hiện tại & Hệ thống báo lỗi "Không thể xóa tài khoản đang hoạt động" & Pass \\
\hline
\end{longtable}

\section*{Test Case – Chức năng Quản lý đơn hàng}

\begin{longtable}{|p{1cm}|p{3cm}|p{4cm}|p{3cm}|p{2cm}|}
\hline
\textbf{TC} & \textbf{Tên kiểm thử} & \textbf{Mô tả kiểm thử} & \textbf{Kết quả mong đợi} & \textbf{Trạng thái} \\
\hline
TC29 & Xem danh sách đơn hàng & Truy cập trang "Quản lý đơn hàng" & Hiển thị danh sách đơn hàng đầy đủ & Pass \\
\hline
TC30 & Hiển thị trạng thái đơn hàng & Quan sát các trạng thái: Chờ xác nhận, Đang giao, Đã hoàn thành, Đã hủy & Trạng thái được hiển thị chính xác bằng nhãn màu rõ ràng & Pass \\
\hline
TC31 & Xem chi tiết đơn hàng & Nhấn vào mã đơn hàng để xem chi tiết sản phẩm, khách hàng & Hiển thị đầy đủ thông tin đơn: tên, sản phẩm đồng hồ, địa chỉ, trạng thái & Pass \\
\hline
TC32 & In đơn hàng & Nhấn nút "In" tại đơn hàng bất kỳ & Hệ thống hiển thị giao diện in đơn hàng & Pass \\
\hline
TC33 & Cập nhật trạng thái đơn hàng & Nhấn chỉnh sửa và chuyển trạng thái từ "Chờ xác nhận" sang "Đang giao" & Trạng thái cập nhật thành công, hiển thị trên giao diện & Pass \\
\hline
TC34 & Trạng thái không hợp lệ & Cố gắng chuyển đơn hàng đã hoàn thành về "Chờ xác nhận" & Hệ thống báo lỗi hoặc không cho phép cập nhật & Pass \\
\hline
TC35 & Hiển thị tổng số đơn hàng & Kiểm tra tổng số đơn hàng hiển thị ở góc trên trang quản lý & Tổng số hiển thị đúng với số lượng đơn trong danh sách & Pass \\
\hline
\end{longtable}

\section*{Test Case – Chức năng Quản lý giỏ hàng}

\begin{longtable}{|p{1cm}|p{3cm}|p{4cm}|p{3cm}|p{2cm}|}
\hline
\textbf{TC} & \textbf{Tên kiểm thử} & \textbf{Mô tả kiểm thử} & \textbf{Kết quả mong đợi} & \textbf{Trạng thái} \\
\hline
TC36 & Cập nhật số lượng sản phẩm đồng hồ & Tăng số lượng sản phẩm đồng hồ từ 1 lên 3 trong giỏ hàng & Tổng tiền được cập nhật chính xác theo số lượng mới & Pass \\
\hline
TC37 & Giảm số lượng xuống 0 & Giảm số lượng sản phẩm đồng hồ xuống 0 & Hệ thống xóa sản phẩm đó khỏi giỏ hàng & Pass \\
\hline
TC38 & Xóa sản phẩm đồng hồ thủ công & Nhấn nút "Xóa" sản phẩm đồng hồ cụ thể trong giỏ hàng & Sản phẩm đồng hồ bị loại bỏ khỏi danh sách giỏ hàng & Pass \\
\hline
TC39 & Áp dụng mã giảm giá hợp lệ & Nhập mã "VC2025" và nhấn "Áp dụng" & Hệ thống hiển thị thông báo áp dụng thành công và giảm giá được tính vào tổng đơn hàng & Pass \\
\hline
TC40 & Áp dụng mã giảm giá không hợp lệ & Nhập mã "ABC123" đã hết hạn & Hệ thống hiển thị thông báo lỗi và không áp dụng giảm giá & Pass \\
\hline
TC41 & Tính tổng đơn hàng có khuyến mãi & Có 2 sản phẩm đồng hồ trong giỏ và áp dụng mã giảm 10\% & Tổng đơn hàng hiển thị chính xác sau khi trừ giảm giá & Pass \\
\hline
TC42 & Tính tổng đơn hàng không có khuyến mãi & Có sản phẩm đồng hồ nhưng không dùng mã giảm giá & Tổng đơn hàng = tổng giá sản phẩm đồng hồ + phí vận chuyển + thuế (nếu có) & Pass \\
\hline
\end{longtable}

\section*{Test Case – Chức năng Đặt hàng và Thanh toán}

\begin{longtable}{|p{1cm}|p{3cm}|p{4cm}|p{3cm}|p{2cm}|}
\hline
\textbf{TC} & \textbf{Tên kiểm thử} & \textbf{Mô tả kiểm thử} & \textbf{Kết quả mong đợi} & \textbf{Trạng thái} \\
\hline
TC43 & Truy cập trang thanh toán & Nhấn "Tiến hành thanh toán" từ trang giỏ hàng & Hiển thị giao diện xác nhận đơn hàng & Pass \\
\hline
TC44 & Nhập thông tin giao hàng hợp lệ & Điền đầy đủ thông tin: họ tên, số điện thoại, địa chỉ & Cho phép nhấn nút "Xác nhận đặt hàng" & Pass \\
\hline
TC45 & Bỏ trống thông tin giao hàng & Bỏ trống địa chỉ hoặc số điện thoại & Hệ thống hiển thị thông báo lỗi và không đặt hàng được & Pass \\
\hline
TC46 & Xác nhận đơn hàng thành công & Nhấn "Xác nhận đặt hàng" sau khi nhập đủ thông tin & Hiển thị thông báo đặt hàng thành công và chuyển đến trang cảm ơn & Pass \\
\hline
TC47 & Kiểm tra trạng thái đơn mới tạo & Vào "Quản lý đơn hàng" và tìm đơn mới đặt & Trạng thái đơn là "Chờ xác nhận" & Pass \\
\hline
TC48 & Thanh toán online thành công & Chọn hình thức thanh toán online và hoàn tất thanh toán & Hệ thống xác nhận thanh toán thành công và trạng thái đơn chuyển sang "Đã thanh toán" & Pass \\
\hline
TC49 & Thanh toán thất bại & Giả lập lỗi thanh toán từ cổng thanh toán & Hiển thị thông báo lỗi và giữ nguyên trạng thái đơn là "Chờ thanh toán" & Pass \\
\hline
\end{longtable}
